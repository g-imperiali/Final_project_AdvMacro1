\documentclass[12pt]{article}%
\usepackage{graphicx} % Required for inserting images
\usepackage[top=1.5in, bottom=1.5in, left=1.5in, right=1.5in]{geometry}
\geometry{left=1in,right=1in,top=1.00in,bottom=1.0in}
\usepackage[singlespacing]{setspace}
\usepackage[bottom]{footmisc}
\usepackage{indentfirst}
\usepackage{hyperref}
\usepackage{amsmath, amssymb}
\usepackage{bbm}
\usepackage{yhmath}
%\setlength\parindent{0pt}
\usepackage{natbib}

\usepackage{xcolor}  % For syntax highlighting
\usepackage{color}
\definecolor{darkgreen}{rgb}{0.0, 0.5, 0.0}
\definecolor{onyx}{HTML}{313638}
\definecolor{lightred}{HTML}{F87575}
\definecolor{melon}{HTML}{FFA9A3}
\definecolor{cornblue}{HTML}{5C95FF}
\definecolor{maize}{HTML}{E7E247}
\definecolor{blendedblue}{rgb}{0.2,0.2,0.7}

\doublespacing

\title{Two-Sided Market Power and Misallocation}
\author{Grace Chuan \& Guglielmo Imperiali d'Afflitto}
\date{July 2025}


\begin{document}

\maketitle



\textcolor{blue}{\textbf{Note for Grace}: In the following, I color with blue the text that what written by you. I have derived all computations on my own, so we can compare our to versions and check for differences}



\section{Motivation}
It is well known that firms with product market power can distort pricing by charging high markups leading to lower social welfare, misallocation, and other negative aggregate macroeconomic effects. Likewise, firms with labor market power lower wages and underhire workers relative to the competitive outcome which result in another form of inefficiency. However, realistically, firms in many key industries and sectors, may jointly hold product and labor market power while facing the same competitors in both markets. For example, Apple and Samsung can be seen as duopolists for smartphones while hiring from the same pool of highly specialized engineers who can mainly only work for one of these two employers. When combining these two frictions, the implications are unclear. On one hand, lower wages as a result of market power may lead to more hired labor and thus greater production on the product side. This is a case in which labor and product market power counteract one another. On the other hand, both forms of market power may exacerbate inefficiency from a combined markup and markdown. We thus investigate this question with a formal analysis building upon 
\citet{Atkeson-Burstein_2008} and incorporating labor market power. 

In our setting, there are heterogeneous finite firms who hold sectoral product and labor market power but are infinitesimal across a continuum of sectors. There is a representative household with standard CES preferences who chooses how much of the aggregate consumption good to consume and how much aggregate labor to supply to all firms. We solve for the industry equilibrium by modifying a well-established algorithm to account for firm-specific wages and markdowns. We then solve for the aggregate equilibrium variables numerically as well. [We find that... MENTION KEY RESULTS] 

\section{Literature Review}
This project sits within the literature on market power 
Our work is closely related to \citet{Deb-et-al_2024} who also solve a model with both labor and product market power in a CES framework to study wage inequality and the skill premium. A key difference is that in their setup, firms hire both high-skilled and low-skilled laborers whose production technologies are CES aggregates of both inputs with some elasticity of substitution. We do not allow for heterogenous laborers and thus keep firm production technologies linear in labor. This is what enables us to modify the original algorithm and produce the equilibrium quantities more efficiently. Furthermore, our relevant outcomes of interest are not the skill premium but of misallocation and how exactly labor market power interacts with product market power. Another subtle, yet inconsequential difference is that \citet{Deb-et-al_2024} allows for an additional layer of granularity where firms also own establishments within their respective sectors while we do not 

\section{Model Setup}

\subsection{Environment}
We consider a closed, static economy with different layers of production: (i) a final good sector where firms compete under perfect competition; and (ii) a continuum of intermediate goods sectors, where firms compete oligopolistically à-la Cournot and sell to the firms in the final good sector. This environment is the same as in \citet{Atkeson-Burstein_2008} and allows us to model variable markups.

Formally, the finite good $Y$ is produced as a CES aggregate over the differentiated intermediate inputs, of which there is a continuum of measure $S$:
\begin{align*}
    Y = \left( \int_{S} S^{- \frac{1}{\theta}} y(s)^{\frac{\theta - 1}{\theta}} ds \right) ^{\frac{\theta}{\theta - 1}}
\end{align*}
while output from each intermediate sector, $y(s)$, is a CES aggregate over the individual outputs from the $I$-many Cournot oligopolists:
\begin{align*}
    y(s) = \left( \sum_{i=1} ^I I^{- \frac{1}{\eta}} y_i(s)^{\frac{\eta - 1}{\eta}} \right) ^{\frac{\eta}{\eta - 1}}
\end{align*}

Furthermore, as in \citet{Deb-et-al_2024}, we assume firms have monopsony power in the labor market and can impose a markdown on wages. Labor is provided by a representative household which has CES preferences over work in the different sectors. The total quantity of work, $L$ is given by:
\begin{align*}
    L &= \left( \int_{S} S^{\frac{1}{\hat{\theta}}} l(s)^{\frac{\hat{\theta} + 1}{\hat{\theta}}} ds \right) ^{\frac{\hat{\theta}}{\hat{\theta} + 1}}
\end{align*}
and within each sector, labor $l(s)$ is allocated to the $I$ different firms according to the following CES aggregate:
\begin{align*}
    l(s) = \left( \sum_{i=1} ^I I^{\frac{1}{\hat{\eta}}} l_i(s)^{\frac{\hat{\eta} + 1}{\hat{\eta}}} \right) ^{\frac{\hat{\eta}}{\hat{\eta} + 1}}
\end{align*}

We assume goods are substitutes, both across sectors as well as across firms in a single sector; furthermore, goods in a single sector are more substitutable than across different sectors. Hence,
\begin{align*}
    \eta > \theta > 1
\end{align*}

Similarly, it is natural to assume that labor is more substitutable between firms in the same sectors than between sectors. So,
\begin{align*}
    \hat{\eta} > \hat{\theta} > 1
\end{align*}



\subsection{Preferences}
The representative household has CRRA utility over consumption and labor subject to a budget constraint:
\begin{align} \label{HH_problem}
    \max_{C,L} U(C, L) &= \frac{C^{1-\sigma}}{1-\sigma} - \frac{L^{1+\phi}}{1+\phi} \\
    P C &= W L + \Pi
\end{align}
$P, W$ are the aggregate CES indices for goods prices and wages, respectively. $\Pi$ is the sum of profits generated by the oligopolistic firms, which are ultimately owned by the representative household.



\subsection{Technology}
In each sector $s$, the $I$-many oligopolists produce according to a linear technology:
\begin{align} \label{firm_technology}
    y_i(s) = z_i(s) l_i(s)
\end{align}

and productivity $z_i$ is iid Pareto with shape parameter $\xi$ within each sector $s$. 



\section{Equilibrium}

We set the consumption aggregate $C$ as the numeraire, so $P = 1$.


\subsection{Household optimality conditions}
The solution to the household problem in \eqref{HH_problem} requires:
\begin{align}
    W = P C^\sigma L^\phi
\end{align}
Moreover, given the CES preferences, the household demand for each good and her allocation of labor to each firm are as follows:
\begin{align}
    y_i(s) &= \frac{1}{I} \frac{1}{S} \left( \frac{p_i(s)}{p(s)} \right)^{-\eta} \left( \frac{p(s)}{1} \right)^{-\theta} Y \label{direct_demand} \\
    l_i(s) &= \frac{1}{I} \frac{1}{S} \left( \frac{W_i(s)}{W(s)} \right)^{\hat{\eta}} \left( \frac{W(s)}{W} \right)^{\hat{\theta}} L \label{direct_laborSupply} 
\end{align}



\subsection{Firm optimality condition}
Each oligopolist/oligopsonist in the intermediate sector sets the good price and the wage to maximize profits:
\begin{align} \label{firm_problem}
    \max_{p_i(s), W_i(s)} & p_i(s) y_i(s) - W_i(s) l_i(s)
\end{align}
subject to
\begin{align}
    p_i(s) &= I^{- \frac{1}{\eta}} J^{-\frac{1}{\theta}} \left( \frac{y_i(s)}{y(s)} \right)^{- \frac{1}{\eta}} \left( \frac{y(s)}{Y} \right)^{- \frac{1}{\theta}} \label{inverse_demand} \\
    W_i(s) &= I^{\frac{1}{\hat{\eta}}} J^{\frac{1}{\hat{\theta}}} \left( \frac{l_i(s)}{l(s)} \right)^{\frac{1}{\hat{\eta}}} \left( \frac{l(s)}{L} \right)^{\frac{1}{\hat{\theta}}} W \label{inverse_laborSupply}
\end{align}
and the technology in \eqref{firm_technology}. This yields:
\begin{align} \label{firm_FOC}
    z_i(s) p_i(s)\left[ 1 - \frac{1}{\eta} + \left( \frac{1}{\eta} - \frac{1}{\theta} \right) \omega_i(s) \right] = W_i(s) \left[ 1 + \frac{1}{\hat{\eta}} + \left(\frac{1}{\hat{\theta}} - \frac{1}{\hat{\eta}}\right) \alpha_i(s) \right]
\end{align}
where $\omega_i(s)$ and $\alpha_i(s)$ denote, respectively, the sales shares and wage bill shares of firm $i$ in sector $s$:
\begin{align*}
    \omega_i(s) & \equiv \frac{p_i(s) y_i(s)}{\sum_i p_i(s) y_i(s)} \\
    \alpha_i(s) & \equiv \frac{W_i(s) l_i(s)}{\sum_i W_i(s) l_i(s)}
\end{align*}
Given \eqref{firm_FOC}, we can define the markup, $\mu_i$, as 
\begin{align} \label{markup_def}
    \mu_i(s) \equiv \left[ 1 - \frac{1}{\eta} + \left( \frac{1}{\eta} - \frac{1}{\theta} \right) \omega_i(s) \right]^{-1}
\end{align}
and the markdown on wages, $\delta_i$, as
\begin{align} \label{markdown_def}
    \delta_i(s) \equiv \left[ 1 + \frac{1}{\hat{\eta}} + \left(\frac{1}{\hat{\theta}} - \frac{1}{\hat{\eta}}\right) \alpha_i(s) \right]
\end{align} 



\subsection{Market clearing}
In equilibrium, given prices on goods and labor, all markets clear -- good and labor markets in the intermediate sectors as well as the good and labor market in the final sector. Hence, the final equilibrium condition we need to state is:\footnote{So far, we have only distinguished between quantities demanded and quantities produced for the final good. For the intermediate goods and for labor, we have used the same notation both for demanded and supplied quantities. This is why we state market clearing only for the final good. }
\begin{align*}
    Y = C
\end{align*}



\subsection{Solution algorithm} \label{section_solutionAlgo}
We propose a solution algorithm for equilibrium allocations, which cannot be found analytically, which is very much in the same spirit as the efficient algorithm proposed in Lecture 9 ``Markups and misallocation'' to solve for the steady state in \citet{Atkeson-Burstein_2008}. 

In section \ref{appendix_sales&wageShares} in the Appendix, we show that the sales share and wage bill share variables can be rewritten as:
\begin{align}
    \omega_i(s) & \equiv \frac{p_i(s) y_i(s)}{\sum_i p_i(s) y_i(s)} = \frac{ [\mu_i(s)\delta_i(s)]^{\frac{\hat{\eta} (1 - \eta)}{\hat{\eta} + \eta}} z_i(s)^{\frac{(\hat{\eta} + 1)(\eta - 1)}{\hat{\eta} + \eta}} }{ \sum_i [\mu_i(s)\delta_i(s)]^{\frac{\hat{\eta} (1 - \eta)}{\hat{\eta} + \eta}} z_i(s)^{\frac{(\hat{\eta} + 1)(\eta - 1)}{\hat{\eta} + \eta}} }
    \label{sales_share} \\
    \alpha_i(s) & \equiv \frac{W_i(s)l_i(s)}{\sum_i W_i(s) l_i(s)} = \frac{ [ \mu_i(s) \delta_i(s) z_i(s)^{-1 + \frac{1}{\eta}} ]^{-\frac{\eta(1+\hat{\eta})}{\hat{\eta} + \eta}} }{ \sum_i [\mu_i(s) \delta_i(s) z_i(s)^{-1 + \frac{1}{\eta}}]^{-\frac{\eta(1+\hat{\eta})}{\hat{\eta} + \eta}} }
    \label{wage_bill_shares}
\end{align}

From this, we proceed as follows: 
\begin{enumerate}
    \item Denote the aggregates $D_1(s)$ and $D_2(s)$ such that 
    \begin{align*}
        \omega_i(s) &= \left[ \frac{ \mu_i(s)\delta_i(s) z_i(s)^{-1-\frac{1}{\hat{\eta}}} }{ D_1(s) } \right]^{\frac{\hat{\eta} (1 - \eta)}{\hat{\eta} + \eta}} \\
        \alpha_i(s) &= \left[ \frac{ \mu_i(s) \delta_i(s) z_i(s)^{-1+\frac{1}{\eta}} }{ D_2(s) } \right]^{-\frac{\eta (1+\hat{\eta})}{\hat{\eta}+\eta}}
    \end{align*}

    \item Define the transformations $f$ and $g$ such that:
    \begin{align*}
        f(i,s) &= \left[ \frac{\delta_i(s) z_i(s)^{-\frac{\hat{\eta} + 1}{\hat{\eta}}} }{ D_1(s) } \right]^{\frac{\hat{\eta} (1 - \eta)}{\hat{\eta} + \eta}} \\
        g(i,s) &= \left[ \frac{\mu_i(s) z_i(s)^{\frac{1-\eta}{\eta}}}{D_2(s)} \right]^{-\frac{\eta (1+\hat{\eta})}{\hat{\eta}+\eta}}
    \end{align*}

    \item Take the following system of $|I| \times |S|$ equations:
    \begin{align}
        1 - \frac{1}{\mu_i(s)} &= \frac{1}{\eta} + (\frac{1}{\theta} - \frac{1}{\eta}) f(i,s) \mu_i(s)^{\frac{\hat{\eta} (1 - \eta)}{\hat{\eta} + \eta}} \label{markup_eq} \\
        \delta_i(s) - 1 &= \frac{1}{\hat{\eta}} + (\frac{1}{\hat{\theta}} - \frac{1}{\hat{\eta}}) g(i,s) \delta_i(s)^{-\frac{\eta (1+\hat{\eta})}{\hat{\eta}+\eta}} \label{markdown_eq}
    \end{align}
    Because $\mu$ and $\delta$ depend one-to-one on $f$ and $g$, we can think of the system above as being determined by the unknowns $\{\mu_i(s), \delta_i(s)\}_{i \in I, s \in S}$. Hence, we can reduce the above system of $|I| \times |S|$ equations to two functional equations per sector, with $\mu \in [{\frac{1}{\eta}}, \frac{1}{\theta}]$ and $\delta \in [\frac{1}{\hat{\eta}}, \frac{1}{\hat{\theta}}]$:
    \begin{align*}
        1 - \frac{1}{\mu} &= \frac{1}{\eta} + (\frac{1}{\theta} - \frac{1}{\eta})f(\mu) \mu^{\frac{\hat{\eta} (1 - \eta)}{\hat{\eta} + \eta}} \\
        \delta - 1 &= \frac{1}{\hat{\eta}} + (\frac{1}{\hat{\theta}} - \frac{1}{\hat{\eta}}) g(\delta) \delta^{-\frac{\eta (1+\hat{\eta})}{\hat{\eta}+\eta}}
    \end{align*}

    \item Computationally, we iterate on the following algorithm:
    \begin{enumerate}
        \item Guess a vector of \textcolor{red}{$\delta^{(0)}$}
        \item Find \textcolor{darkgreen}{$D_1$} that solves Eq. \eqref{markup_eq}
        % \begin{align}
        %     \textcolor{darkgreen}{D_1(s)} = (\sum_i^{I} [\textcolor{red}{\delta_i(s)} z_i(s)^{-\frac{1}{ \hat{\eta}}-1}U((\frac{\textcolor{red}{\delta_i(s)} z_i(s)^{-\frac{1}{\hat{\eta}}-1}}{\textcolor{darkgreen}{D_1(s)}})^{\frac{\hat{\eta}(1-\eta)}{\eta + \hat{\eta}}})]^{\frac{\hat{\eta}(1-\eta)}{\eta+\hat{\eta}}})^{\frac{\eta+\hat{\eta}  }{\hat{\eta}(1-\eta)}}
        % \end{align}
        
        \item Having obtained \textcolor{darkgreen}{$D_1$}, compute $f$ and  $\mu$.
        
        \item Use $\mu$ to solve for \textcolor{melon}{$D_2$}. 
    %     \begin{align}
    %          \textcolor{melon}{D_2(s)} = (\sum_i^{n(s)} [z_i(s)^{\frac{1}{\eta}-1}\Delta((\frac{\textcolor{blue}{\mu_i(s)} z_i(s)^{\frac{1}{\eta}-1}}{\textcolor{melon}{D_2(s)}})^{-\frac{\eta(\hat{\eta}+1)}{\eta+\hat{\eta}}})\textcolor{blue}{\mu_i(s)}]^{-\frac{\eta(\hat{\eta}+1)}{\eta+\hat{\eta}}})^{-\frac{\eta+\hat{\eta}}{\eta(\hat{\eta}+1)}}
    %     \end{align}
        
        \item With \textcolor{melon}{$D_2$} compute $g$ and $\delta^{(1)}$
        
        \item Compare resulting $\delta^{(1)}$ with initial guess \textcolor{red}{$\delta^{(0)}$} 
        
        \item Update guess with $\delta^{(1)}$  and iterate (a) to (g) until $\delta$ converges    
    \end{enumerate}
\end{enumerate}

\section{Aggregation}
Next, we show that the algorithm we propose is `efficient' in the sense that the equilibrium allocations can be retrieved from the quantities $\{z_i(s), \mu_i(s), \delta_i(s), \omega_i(s), \alpha_i(s)\}_{i,s}$ obtained in section \ref{section_solutionAlgo}.

First, let us define the following aggregates:
\begin{align} \label{aggregates_definitions}
    \begin{array}{lllll}
        & z(s) \equiv \frac{y(s)}{l(s)}, 
        && \mu(s) \delta(s) \equiv \frac{p(s) y(s)}{W(s) l(s)}, \quad
        \omega(s) \equiv \frac{p(s) y(s)}{P Y}, \quad
        \alpha(s) \equiv \frac{W(s) l(s)}{W L}, \\
        & Z \equiv \frac{Y}{L}, && M \Delta \equiv \frac{P Y}{W L} 
    \end{array} 
\end{align}
Then, in section \ref{appendix_aggregates} in the Appendix we show how to get the formulas for the aggregate allocations:
\begin{align*}
    \mu(s) \delta(s) & = \left( \sum_i \frac{\omega_i(s)}{\mu_i(s) \delta_i(s)} \right)^{-1} \\
    \frac{p_i(s)}{p(s)} &= \left[ I \frac{\mu_i(s) \delta_i(s)}{\mu(s) \delta(s)} \alpha_i(s) \right]^{\frac{1}{1-\eta}} 
    \implies z(s) = \frac{I}{\sum_i \frac{1}{z_i(s)} \left[ \frac{p_i(s)}{p(s)} \right]^{-\eta}} = \frac{I^{\frac{1}{1-\eta}}}{\sum_i \frac{1}{z_i(s)} \left[ \frac{\mu_i(s) \delta_i(s)}{\mu(s) \delta(s)} \alpha_i(s) \right]^{-\frac{\eta}{1-\eta}}} \\
    W(s) &= \frac{\mu(s) \delta(s) z(s)^{-1}}{ \left[ \frac{1}{I} \sum_i \left(\frac{\mu_i(s) \delta_i(s)}{z_i(s)} (I \alpha_i(s))^{\frac{1}{1+\hat{\eta}}} \right) ^{1-\eta} \right]^\frac{1}{1-\eta}}\\
    \omega(s) &= \frac{1}{S} \left( \frac{\mu(s) \delta(s) W(s)}{z(s)} \right)^{1-\theta}  \\
    M \Delta &= \left( \int_S \frac{\omega(s)}{\mu(s)\delta(s)} ds \right)^{-1} \\
    p(s) &= \left( \omega(s) S \right)^{\frac{1}{1-\theta}} \\
    Z &= \left( \frac{1}{S} \int_S \frac{1}{Z(s)} p(s)^{-\theta} \right)^{-1} \\
    W &= \frac{Z}{M \Delta} \\
    L &= \left( W Z^{-\sigma} \right) ^{\frac{1}{\sigma + \phi}}
\end{align*}
and from this all the remaining variables can be obtained trivially. Therefore, there is no need to iterate on good excess demand to obtain the aggregate allocations.

\section{Results}

\subsection{Markups and market share}
% Graph like in lecture: plot 2 lines - 1) omega against mu our setting and 2) omega against mu w/o labor market power

\subsection{Markdowns and labor share}

% Graph alpha against delta 

\subsection{Concentration}

% top 1/3 of table from lecture with more percentiles
% histogram

% compare with class pset results

% correlation between firm  concentration and markup scatterplot of HHI and mu

% correlation between firm markup and firm markdowns

\subsection{Cost-weighted markups and markdowns}


\subsection{Aggregate statistics}

\subsection{Welfare analysis}



\newpage

% \textcolor{magenta}{\section{Aggregation}}

% Proposed approach:
% \begin{enumerate}
%     \item Calculate the demand and supply elasticities using calculated markups
%     \item Get $p_i(s)$ and $W_i(s)$ from elasticities
%     \item Aggregate $p_i(s)$ to get $p(s)$.
%     \item Aggregate $W_i(s)$ to get $W(s)$ and $W$
%     \item Calculate $y_i(s)$ and $l_i(s)$ through iteration: 
%     \begin{enumerate}
%         \item Guess $Y$ old
%         \item $y_i(s) = (\frac{p_i(s)}{p(s)}\frac{p(s)}{1})^{-\theta}Y$
%         \item $l_i(s) = \frac{y_i(s)}{z_i(s)}$
%         \item Aggregate $l_i(s)$ to $L$
%         \item Calculate $Y$ new via HH condition $Y^\sigma = \frac{W}{L^\phi}$
%         \item Iterate until Y new and Y old same
%     \end{enumerate}
%     \item Get aggregates $W$, $Y=C$, $L$
%     \item Calculate $Z$ via method in presentation using $z_i(s)$ and $\omega_i(s)$
% \end{enumerate}




%%%%%%%%%%%%%%%%%%%%%%%%%%%%%%%%%%%%%%%%%%%%%%%%%%%%%%%%%%%%%%%%%%%%%%%%%%%
% Bibliography
%%%%%%%%%%%%%%%%%%%%%%%%%%%%%%%%%%%%%%%%%%%%%%%%%%%%%%%%%%%%%%%%%%%%%%%%%%%
\bibliography{references.bib}
\bibliographystyle{apalike}



%%%%%%%%%%%%%%%%%%%%%%%%%%%%%%%%%%%%%%%%%%%%%%%%%%%%%%%%%%%%%%%%%%%%%%%%%%%
% Appendix
%%%%%%%%%%%%%%%%%%%%%%%%%%%%%%%%%%%%%%%%%%%%%%%%%%%%%%%%%%%%%%%%%%%%%%%%%%%
\newpage
\appendix
{\Large \textbf{Appendix}}
\section{Equations}

\subsection{Manipulation of sales share and wage bill share} \label{appendix_sales&wageShares}
Consider the direct goods demand, direct labor supply, indirect goods demand, and indirect labor supply in \eqref{direct_demand}, \eqref{direct_laborSupply}, \eqref{inverse_demand}, \eqref{inverse_laborSupply}. Define the following sectoral aggregates:
\begin{align*}
    \mathcal{K}_C(s) & \equiv \frac{1}{I} \frac{1}{J} p(s)^{\eta - \theta} Y \\
    \mathcal{H}_P(s) & \equiv I^{- \frac{1}{\eta}} J^{- \frac{1}{\theta}} y(s)^{\frac{1}{\eta} - \frac{1}{\theta}} Y^{\frac{1}{\theta}}  \\
    \mathcal{K}_L(s) & \equiv \frac{1}{I} \frac{1}{J} W(s)^{\hat{\theta} - \hat{\eta}} W^{-\hat{\theta}} L \\
    \mathcal{H}_W(s) & \equiv I^{\frac{1}{\hat{\eta}}} J^{\frac{1}{\hat{\theta}}} l(s)^{\frac{1}{\hat{\theta}} - \frac{1}{\hat{\eta}}} L^{- \frac{1}{\hat{\theta}}} W
\end{align*}

Then, we can write the equilibrium wage and price, from \eqref{firm_FOC}, as:
\begin{align*}
    W_i(s) &= \frac{z_i(s) p_i(s)}{\mu_i(s) \delta_i(s)} = \frac{z_i(s)}{\mu_i(s) \delta_i(s)} y_i(s)^{-\frac{1}{\eta}} \mathcal{H}_P(s) = \frac{z_i(s) \mathcal{H}_P(s)}{\mu_i(s) \delta_i(s)} (z_i(s)l_i(s))^{-\frac{1}{\eta}}  \\
    & = \frac{z_i(s)^{\frac{\eta - 1}{\eta}} \mathcal{H}_P(s)}{\mu_i(s) \delta_i(s)} W_i(s)^{-\frac{\hat{\eta}}{\eta}} \mathcal{K}_L(s)^{-\frac{1}{\eta}} \\
    \iff W_i(s) &= \left[ \frac{z_i(s)^{\frac{\eta-1}{\eta}} \mathcal{H}_P(s) \mathcal{K}_L(s)^{-\frac{1}{\eta}}}{\mu_i(s) \delta_i(s)} \right]^{\frac{\eta}{\eta + \hat{\eta}}} 
\end{align*}
and, analogously,
\begin{align*}
    p_i(s) = \left[ \frac{\mu_i(s) \delta_i(s) \mathcal{H}_W(s) \mathcal{K}_C(s)^{\frac{1}{\hat{\eta}}}}{z_i(s) ^{\frac{\hat{\eta} + 1}{\hat{\eta}}}} \right]^{\frac{\hat{\eta}}{\hat{\eta} + \eta}}
\end{align*}
From these, we get the following expressions for the sales share and the wage bill share, respectively:
\begin{align*}
    \omega_i(s) & \equiv \frac{p_i(s) y_i(s)}{\sum_i p_i(s) y_i(s)} = \frac{ [\mu_i(s)\delta_i(s)]^{\frac{\hat{\eta} (1 - \eta)}{\hat{\eta} + \eta}} z_i(s)^{\frac{(\hat{\eta} + 1)(\eta - 1)}{\hat{\eta} + \eta}} }{ \sum_i [\mu_i(s)\delta_i(s)]^{\frac{\hat{\eta} (1 - \eta)}{\hat{\eta} + \eta}} z_i(s)^{\frac{(\hat{\eta} + 1)(\eta - 1)}{\hat{\eta} + \eta}} }\\
    \alpha_i(s) & \equiv \frac{W_i(s)l_i(s)}{\sum_i W_i(s) l_i(s)} = \frac{ [\mu_i(s) \delta_i(s)] ^{-\frac{\eta (1+\hat{\eta})}{\hat{\eta}+\eta}} z_i(s)^{\frac{(\hat{\eta} + 1)(\eta-1)}{\hat{\eta}+\eta}} }{ \sum_i [\mu_i(s) \delta_i(s)] ^{-\frac{\eta (1+\hat{\eta})}{\hat{\eta}+\eta}} z_i(s)^{\frac{(\hat{\eta} + 1)(\eta-1)}{\hat{\eta}+\eta}} }
\end{align*}

\subsection{Obtaining aggregate allocations} \label{appendix_aggregates}
Given the vectors $\{z_i(s), \omega_i(s), \mu_i(s), \alpha_i(s), \delta_i(s)\}_{i,s}$, we can obtain the sectoral aggregates and the economy-wide aggregates as follows.


\begin{enumerate}
    \item From the FOC of the intermediate firms in \eqref{firm_FOC} we have $\frac{p_i(s) y_i(s)}{\mu_i(s) \delta_i(s)} = W_i(s) l_i(s)$. By definition of sectoral markup and markdown, we have $\frac{p(s) y(s)}{\mu(s) \delta(s)} = W(s) l(s)$. Because $W(s) l(s) = \sum_i W_i(s) l_i(s)$, then we get $\frac{p(s) y(s)}{\mu(s) \delta(s)} = \sum_i \frac{p_i(s) y_i(s)}{\mu_i(s) \delta_i(s)}$, i.e., $\mu(s) \delta(s) = \left( \sum_i \frac{\omega_i(s)}{\mu_i(s) \delta_i(s)} \right) ^{-1}$.

    \item Because $\frac{p_i(s)}{p(s)} = \frac{\mu_i(s) \delta_i(s) W_i(s) z_i(s)^{-1}}{\mu(s) \delta(s) W(s) z(s)^{-1}}$, then $\frac{p_i(s)}{p(s)}\frac{y_i(s)}{y(s)} = \frac{\mu_i(s) \delta_i(s) W_i(s) l_i(s)}{\mu(s) \delta(s) W(s) l(s)}$, so 
    $$\frac{p_i(s)}{p(s)} = \left( I \frac{\mu_i(s) \delta_i(s)}{\mu(s) \delta(s)} \alpha_i(s) \right)^{\frac{1}{1-\eta}}.$$ Therefore, we can compute $z(s)$, given that $z(s) = \left( \frac{1}{I} \sum_i z_i(s)^{-1 -\frac{1}{\hat{\eta}}} \left( \frac{p_i(s)}{p(s)} \right)^{-\eta \frac{\hat{\eta}+1}{\hat{\eta}}} \right)^{-\frac{\hat{\eta}}{\hat{\eta}+1}}$.

    \item  Next we show how to find $\omega(s)$.
    
    $\omega(s) \equiv \frac{p(s) y(s)}{1 \cdot Y} = \left( \frac{p(s)}{1} \right)^{1-\theta} \frac{1}{S} = \frac{1}{S} \left( \frac{\mu(s) \delta(s) W(s)}{z(s)} \right)^{1-\theta}$. So, $W(s)^{\theta-1} \omega(s) = \frac{1}{S} \left( \frac{\mu(s) \delta(s) }{z(s)} \right)^{1-\theta}$. But then,
    \begin{align*}
        & W(s)^{\theta-1} \omega(s) = W(s)^{\theta-1} \frac{1}{S} p(s)^{1-\theta} = W(s)^{\theta-1} \frac{1}{S} \left[ \frac{1}{I} \sum_i p_i(s)^{1-\eta} \right]^{\frac{1-\theta}{1-\eta}} \\
        & = \frac{1}{S} \left[ \frac{1}{I} \sum_i \left( \frac{\mu_i(s) \delta_i(s) W_i(s)}{W(s) z_i(s)} \right)^{1-\eta} \right]^{\frac{1-\theta}{1-\eta}} = \frac{1}{S} \left[ \frac{1}{I} \sum_i \left( \frac{\mu_i(s) \delta_i(s) W_i(s) l_i(s)}{W(s) y_i(s)} \right)^{1-\eta} \right]^{\frac{1-\theta}{1-\eta}} \\
        & = \frac{1}{S} \left[ \frac{1}{I} \sum_i \left( \frac{\mu_i(s) \delta_i(s) W_i(s) l_i(s)}{W(s) \left(\frac{p_i(s)}{p(s)}\right)^{-\eta} \frac{y(s)}{I}} \right)^{1-\eta} \right]^{\frac{1-\theta}{1-\eta}} = \frac{1}{S} \left[ \frac{1}{I} \sum_i \left( \frac{\mu_i(s) \delta_i(s) W_i(s) l_i(s)}{W(s) \left(\frac{p_i(s)}{p(s)}\right)^{-\eta} \frac{z(s) l(s)}{I}} \right)^{1-\eta} \right]^{\frac{1-\theta}{1-\eta}} \\
        & = \frac{1}{S} \left[ \frac{1}{I} \sum_i \left( \frac{\mu_i(s) \delta_i(s)}{ \left(\frac{p_i(s)}{p(s)}\right)^{-\eta} z(s)} \alpha_i(s) I \right)^{1-\eta} \right]^{\frac{1-\theta}{1-\eta}} 
    \end{align*}
    \textcolor{red}{We can only get $W^{\theta-1} \omega(s)$, but then we cannot separate the aggregate wage $W$ from $\omega(s)$. If we found a way to separate the 2, we would be golden}\\
    \textcolor{magenta}{Can we do some iteration on $W$? We start with a guess $W$ and check later conditions you have.}

    \item By definition of the aggregate markup and markdown in \eqref{aggregates_definitions}, we have:
    \begin{align*}
        \frac{P Y}{M \Delta} = W L = \int_S W(s) l(s) ds \iff M \Delta = \left( \int_S \frac{p(s) y(s)}{\mu(s) \delta(s)} \frac{1}{P Y} ds \right)^{-1} = \left( \int_S \frac{\omega(s)}{\mu(s) \delta(s)} ds \right)^{-1}
    \end{align*}

    \item We can recover $Z$ from $Z = \left(\frac{1}{S} \int_S z(s)^{-\frac{\hat{\theta} + 1}{\hat{\theta}}} p(s)^{-\theta \frac{\hat{\theta}+1}{\hat{\theta}}} ds \right)^{-\frac{\hat{\theta}}{\hat{\theta} + 1}}$ given that we have both $z(s)$ and $p(s)$.
    % \textcolor{magenta}{\item Start with 3 equations:
    % \begin{align}
    %     \frac{Y}{M\Delta} = WL\\
    %     Y^\sigma = \frac{W}{L^\phi}\\
    %     Y = ZL
    % \end{align}
    % Combine equations (19) and (20) to get,
    % \begin{align}
    %     W = Z^\sigma L^{\sigma + \phi}
    % \end{align}
    % Next replace $Y$ and $W$ in equation (18) to get $L$ in terms of $Z$ and $M\Delta$,
    % \begin{align}
    %     \hat{L} = (\frac{Z^{1-\sigma}}{M\Delta})^{\frac{1}{\sigma + \phi}}
    % \end{align}
    % Using equation (19) again, we solve for $W$
    % \begin{align}
    %     (Z\hat{L})^\sigma &= \frac{W}{\hat{L}^{\phi}}\\
    %     \implies W &= Z^\sigma \hat{L}^{\sigma + \phi}\\
    %     &=  Z^\sigma(\frac{Z^{1-\sigma}}{M\Delta})^{\frac{\sigma + \phi}{\sigma + \phi}}\\
    %     &= \frac{Z}{M\Delta}
    % \end{align}
    % }
    \item All the other aggregates follow easily: $W = \frac{P Y}{M \Delta}$, $L = Z^{\frac{-\sigma}{\sigma + \phi}} W^{\frac{1}{\sigma+\phi}}$, $C = Z L$.

\end{enumerate}









\newpage

{\color{blue}
% 


\section{Model Setup}
[\textit{The following has been written and derived by Grace -- July 16, 2025}]
\subsection{Environment}

\subsection{Preferences}

CHANGE TO STANDARD REPRESENTATIVE HOUSEHOLD\\

The representative household chooses consumption and its supply of high-skilled and low-skilled labor. Let $\eta$ be the elasticity of goods $i$ within a market and $\theta$ the elasticity of goods across markets $j$. $\eta > \theta$ because goods within the market are more substitutable than goods across. \\

For the labor market, let $\{\hat{\eta}_L, \hat{\eta}_H\}$ and $\{\hat{\theta}_L,\hat{\theta}_H\} $ be the elasticities of substitution for low-skilled and high-skilled laborers respectively within and across markets. Similarly, $\hat{\eta}_S > \hat{\theta}_S, \forall S \in \{L,H\}$.

\begin{align}
    \text{max}_{C_{inj}, L_{inj}, H_{inj}} \hspace{0.2 cm} C - \frac{1}{\phi_L^{-\frac{1}{\phi_L}}}\frac{L^{\frac{\phi_L +1}{\phi_L}}}{\frac{\phi_L +1}{\phi_L}} - \frac{1}{\phi_H^{-\frac{1}{\phi_H}}}\frac{H^{\frac{\phi_H +1}{\phi_H}}}{\frac{\phi_H +1}{\phi_H}} \\
    \text{s.t.} \hspace{0.2 cm} PC =LW_L+HW_H+\Pi
\end{align}

where $C, H, and L$ are the CES indices for aggregate consumption and employment of
high- and low-skilled workers. $P, W_H,$ and $ W_L $are the CES aggregated indices for prices and wages. 

\begin{align}
    C = (\int_j J^{-\frac{1}{\theta}} C_j^{\frac{\theta-1}{\theta}}dj)^{\frac{\theta}{\theta-1}} \\
    C_j = (\sum_i I^{-\frac{1}{\eta}} C_{inj}^{\frac{\eta-1}{\eta}})^{\frac{\eta}{\eta-1}} \\
    S = (\int_j J^{\frac{1}{\hat{\theta}_S}} C_j^{\frac{\hat{\theta}_S-1}{\hat{\theta}_S}}dj)^{\frac{\hat{\theta}_S}{\hat{\theta}_S-1}} \\
    S_j = (\sum_i I^{\frac{1}{\hat{\eta_S}}} C_{inj}^{\frac{\hat{\eta_S}+1}{\hat{\eta_S}}})^{\frac{\hat{\eta_S}}{\hat{\eta_S}+1}} \\
    , \forall S \in \{H,L\}
\end{align}


\subsection{Technology}

% \begin{align}
%     Y_{inj} = [(A_{Linj}L_{inj})^{\frac{\sigma-1}{\sigma}} + (A_{Hinj}H_{inj})^{\frac{\sigma-1}{\sigma}}]^{\frac{\sigma}{\sigma-1}}
% \end{align}

\begin{align}
    y_i(s) = z_i(s) l_i(s)
\end{align}

\section{Key Equations}

\subsection{Firm demand and supply curves}

\begin{align}
p_i(s) = (\frac{y_i(s)}{y(s)})^{-\frac{1}{\eta}}(\frac{y(s)}{Y})^{-\frac{1}{\theta}}P\\
y_i(s) = (\frac{p_i(s)}{p(s)})^{-\eta}(\frac{p(s)}{P})^{-\theta}Y\\
    W_i(s) = (\frac{L_i(s)}{L(s)})^{\frac{1}{\hat{\eta}}}(\frac{L(s)}{L})^{\frac{1}{\hat{\theta}}} W\\
    L_i(s) = (\frac{W_i(s)}{W(s)})^{\hat{\eta}}(\frac{W(s)}{W})^{\hat{\theta}} L\\
     % W^H_i(s) = (\frac{H_i(s)}{H(s)})^{\frac{1}{\hat{\eta}_H}}(\frac{H(s)}{H})^{\frac{1}{\hat{\theta}_H}} W_H\\
     % H_i(s) = (\frac{W^H_i(s)}{W^H(s)})^{\hat{\eta}_H}(\frac{W^H(s)}{W^H})^{\hat{\theta}_H} H\\
\end{align}

\subsection{Pricing}

\begin{align}
    p_i(s) = \frac{\epsilon_i^P(s)}{\epsilon_i^P(s)-1} \frac{W }{z_i(s)}(1+\frac{1}{\epsilon_i^S(s)})
\end{align}
s.t. $\epsilon_i^P(s)$ is the demand elasticity and $\epsilon_i^S(s)$ is the supply elasticity of laborer $S \in \{L,H\}$.\\

Demand elasticity and labor elasticity are decreasing in market and labor shares,\\
\begin{align}
    \epsilon_i^P(s) = (\omega_i(s)\frac{1}{\theta} + (1-\omega_i(s))\frac{1}{\eta})^{-1}\\
    \epsilon^W_i(s) =  (\alpha_i(s)\frac{1}{\hat{\theta}} + (1-\alpha_i(s))\frac{1}{\hat{\eta}})^{-1}
\end{align}

\subsection{Market share and Labor share}

Let $W^L_i(s) = L_i(s)^{\frac{1}{\hat{\eta}}}\underbrace{\frac{1}{L(s)}^{\frac{1}{\hat{\eta}}}(\frac{L(s)}{L})^{\frac{1}{\hat{\theta}}} W_L}_{C_W(s)} = L_i(s)^{\frac{1}{\hat{\eta}}} C_W(s)$ and $y_i(s) = p_i(s)^{-\eta}\underbrace{\frac{1}{p(s)}^{\eta}(\frac{p(s)}{P})^{-\theta}Y}_{C_y(s)}\\$

\begin{align}
    p_i(s) &= W_i(s)\delta_i(s)\frac{\mu_i(s)}{z_i(s)}\\
    %y_i(s) p_i(s) &= W_i(s)\delta_i(s)\frac{\mu_i(s)}{z_i(s)}y_i(s)\\
    &= L_i(s)^{\frac{1}{\hat{\eta}}} C_W(s)\delta_i(s)\frac{\mu_i(s)}{z_i(s)}\\
    &= (\frac{y_i(s)}{z_i(s)})^{\frac{1}{\hat{\eta}}} C_W(s)\delta_i(s)\frac{\mu_i(s)}{z_i(s)}\\
    &= C_W(s) C_y(s)^{\frac{1}{\hat{\eta}}} (\frac{p_i(s)^{-\eta}}{z_i(s)})^{\frac{1}{\hat{\eta}}} \delta_i(s)\frac{\mu_i(s)}{z_i(s)}\\
    p_i(s)^{1+\frac{\eta}{ \hat{\eta}}} &=  C_W(s) C_y(s)^{\frac{1}{\hat{\eta}}} \delta_i(s)z_i(s)^{-\frac{1}{ \hat{\eta}}-1}\mu_i(s)\\
    p_i(s) &=  [C_W(s) C_y(s)^{\frac{1}{\hat{\eta}}} \delta_i(s)z_i(s)^{-\frac{1}{ \hat{\eta}}-1}\mu_i(s)]^{\frac{ \hat{\eta}}{\eta+\hat{\eta}}}\\
    y_i(s)p_i(s) &= p_i(s)^{1-\eta} = [C_W(s) C_y(s)^{\frac{1}{\hat{\eta}}} \delta_i(s)z_i(s)^{-\frac{1}{ \hat{\eta}}-1}\mu_i(s)]^{\frac{\hat{\eta}(1-\eta)}{\eta+\hat{\eta}}}\\
    \sum_i^{n(s)} y_i(s) p_i(s) &= (C_W(s) C_y(s)^{\frac{1}{\hat{\eta}}})^{\frac{\hat{\eta}(1-\eta)}{\eta+\hat{\eta}}}\sum_i^{n(s)} [\delta_i(s)z_i(s)^{-\frac{1}{ \hat{\eta}}-1}\mu_i(s)]^{\frac{\hat{\eta}(1-\eta)}{\eta+\hat{\eta}}}\\
    \omega_i(s) &= \frac{p_i(s) y_i(s)}{\sum_i^{n(s)} y_i(s) p_i(s)} =  \frac{[\delta_i(s)z_i(s)^{-\frac{1}{ \hat{\eta}}-1}\mu_i(s)]^{\frac{\hat{\eta}(1-\eta)}{\eta+\hat{\eta}}}}{\sum_i^{n(s)} [\delta_i(s)z_i(s)^{-\frac{1}{ \hat{\eta}}-1}\mu_i(s)]^{\frac{\hat{\eta}(1-\eta)}{\eta+\hat{\eta}}}}
\end{align}
\newpage
Let $p_i(s) = y_i(s)^{-\frac{1}{\eta}}\underbrace{(\frac{1}{y(s)})^{-\frac{1}{\eta}}(\frac{y(s)}{Y})^{-\frac{1}{\theta}}P}_{C_p(s)}$ and $L_i(s) = W_i(s)^{\hat{\eta}} C_L(s)$

\begin{align}
    W_i(s) &= \frac{p_i(s)z_i(s)}{\delta_i(s)\mu_i(s)}\\
    &= y_i(s)^{-\frac{1}{\eta}} \frac{C_p(s) z_i(s)}{\delta_i(s)\mu_i(s)}\\
    &= (z_i(s)L_i(s))^{-\frac{1}{\eta}}\frac{C_p(s) z_i(s)}{\delta_i(s)\mu_i(s)}\\
    &= (z_i(s) W_i(s)^{\hat{\eta}} C_L(s))^{-\frac{1}{\eta}}\frac{C_p(s) z_i(s)}{\delta_i(s)\mu_i(s)}\\
    W_i(s)^{1+\frac{\hat{\eta}}{\eta}} &= z_i(s)^{-\frac{1}{\eta}+1}C_L(s)^{-\frac{1}{\eta}}C_p(s) \delta_i(s)^{-1}\mu_i(s)^{-1}\\
    W_i(s) &= [z_i(s)^{-\frac{1}{\eta}+1}C_L(s)^{-\frac{1}{\eta}}C_p(s) \delta_i(s)^{-1}\mu_i(s)^{-1}]^{\frac{\eta}{\eta+\hat{\eta}}}\\
    W_i(s) L_i(s) &= W_i(s)^{1+\hat{\eta}}= [z_i(s)^{-\frac{1}{\eta}+1}C_L(s)^{-\frac{1}{\eta}}C_p(s) \delta_i(s)^{-1}\mu_i(s)^{-1}]^{\frac{\eta(\hat{\eta}+1)}{\eta+\hat{\eta}}}\\
    \sum_i^{n(s)}W_i(s) L_i(s) &= (C_L(s)^{-\frac{1}{\eta}}C_p(s))^{\frac{\eta(\hat{\eta}+1)}{\eta+\hat{\eta}}} \sum_i^{n(s)} [z_i(s)^{-\frac{1}{\eta}+1}\delta_i(s)^{-1}\mu_i(s)^{-1}]^{\frac{\eta(\hat{\eta}+1)}{\eta+\hat{\eta}}}  \\
    &= (C_L(s)^{-\frac{1}{\eta}}C_p(s))^{\frac{\eta(\hat{\eta}+1)}{\eta+\hat{\eta}}} \sum_i^{n(s)} [z_i(s)^{\frac{1}{\eta}-1}\delta_i(s)\mu_i(s)]^{-\frac{\eta(\hat{\eta}+1)}{\eta+\hat{\eta}}}\\
    \alpha_i(s) &= \frac{W_i(s) L_i(s)}{\sum_i^{n(s)}W_i(s) L_i(s)}=\frac{[z_i(s)^{\frac{1}{\eta}-1}\delta_i(s)\mu_i(s)]^{-\frac{\eta(\hat{\eta}+1)}{\eta+\hat{\eta}}}}{\sum_i^{n(s)} [z_i(s)^{\frac{1}{\eta}-1}\delta_i(s)\mu_i(s)]^{-\frac{\eta(\hat{\eta}+1)}{\eta+\hat{\eta}}}}
\end{align}

% \begin{align}
%     \omega_i(s) = (\frac{p_i(s)}{p(s)})^{1-\eta} = \frac{[W^S_i(s)\delta^S_i(s) \mu_i(s)]^{1-\eta}z_i(s)^{\eta-1}}{\sum_{i}^{n(s)} [W^S_i(s)\delta^S_i(s) \mu_i(s)]^{1-\eta}z_i(s)^{\eta-1}}\\
%     \alpha_i(s) = (\frac{W^S_i(s)}{W^S(s)})^{1+\hat{\eta}} = \frac{(p_i(s)z_i(s))^{1+\hat{\eta}}(\delta_i(s)\mu_i(s))^{\hat{\eta}-1}}{\sum_{i}^{n(s)}(p_i(s)z_i(s))^{1+\hat{\eta}}(\delta_i(s)\mu_i(s))^{\hat{\eta}-1}}
% \end{align}

%Goal: get 2n equations and 2n unknowns to adapt the algorithm, we have markdowns as well now


\section{Solving for Industry Equilibrium}

\begin{align}
    1-\frac{1}{\mu_i(s)} = \frac{1}{\epsilon^P_i(s)} = \frac{1}{\eta} + (\frac{1}{\theta} - \frac{1}{\eta} )\omega_i(s)\\
    \delta_i(s) - 1 = \frac{1}{\epsilon^W_i(s)} = \frac{1}{\hat{\eta}} + (\frac{1}{\hat{\theta}} - \frac{1}{\hat{\eta}} )\alpha_i(s)
\end{align}

\subsection{Proposed Algorithm}
\begin{enumerate}
    \item Define aggregates $D_1(s)$ and $D_2(s)$ to go with equations (27) and (37)
    \begin{align}
        \omega_i(s) = \frac{[\delta_i(s)z_i(s)^{-\frac{1}{ \hat{\eta}}-1}\mu_i(s)]^{\frac{\hat{\eta}(1-\eta)}{\eta+\hat{\eta}}}}{\sum_i^{n(s)} [\delta_i(s)z_i(s)^{-\frac{1}{ \hat{\eta}}-1}\mu_i(s)]^{\frac{\hat{\eta}(1-\eta)}{\eta+\hat{\eta}}}}\\
        D_1(s) = (\sum_i^{n(s)} [\delta_i(s)z_i(s)^{-\frac{1}{ \hat{\eta}}-1}\mu_i(s)]^{\frac{\hat{\eta}(1-\eta)}{\eta+\hat{\eta}}})^{\frac{\eta+\hat{\eta}  }{\hat{\eta}(1-\eta)}}\\
        \alpha_i(s)  =\frac{[z_i(s)^{\frac{1}{\eta}-1}\delta_i(s)\mu_i(s)]^{-\frac{\eta(\hat{\eta}+1)}{\eta+\hat{\eta}}}}{\sum_i^{n(s)} [z_i(s)^{\frac{1}{\eta}-1}\delta_i(s)\mu_i(s)]^{-\frac{\eta(\hat{\eta}+1)}{\eta+\hat{\eta}}}}\\
        D_2(s) = (\sum_i^{n(s)} [z_i(s)^{\frac{1}{\eta}-1}\delta_i(s)\mu_i(s)]^{-\frac{\eta(\hat{\eta}+1)}{\eta+\hat{\eta}}})^{-\frac{\eta+\hat{\eta}}{\eta(\hat{\eta}+1)}}
    \end{align}
    \item Define $x_i(s)$ and $g_i(s)$ s.t. $x_i(s)$ is a function of $D_1(s)$ and $\delta_i(s)$; $g_i(s)$ is a function of $D_2(s)$ and $\mu_i(s)$
    \begin{align}
        x_i(s) = (\frac{\delta_i(s) z_i(s)^{-\frac{1}{\hat{\eta}}-1}}{D_1(s)})^{\frac{\hat{\eta}(1-\eta)}{\eta + \hat{\eta}}}\\
        g_i(s) = (\frac{\mu_i(s) z_i(s)^{\frac{1}{\eta}-1}}{D_2(s)})^{-\frac{\eta(\hat{\eta}+1)}{\eta+\hat{\eta}}}
    \end{align}
    \item Solve for $x(\mu)$ for $\mu \in [a_1,b_1]$ interpolate retrieving function $U(x)$
    \begin{align}
    1- \frac{1}{\mu_i(s)}   = \frac{1}{\eta} + (\frac{1}{\theta} - \frac{1}{\eta} )\mu_i(s)^{\frac{\hat{\eta}(1-\eta)}{\eta + \hat{\eta}}} x_i(s)
    \end{align}
    \item Solve for $g(\delta)$ for $\delta \in [a_2,b_2]$ interpolate retrieving function $\Delta(g)$
    \begin{align}
        \delta_i(s)-1   = \frac{1}{\hat{\eta}} + (\frac{1}{\hat{\theta}} - \frac{1}{\hat{\eta}} )\delta_i(s)^{-\frac{\eta(\hat{\eta}+1)}{\eta+\hat{\eta}}} g_i(s)
    \end{align}
    \item Iteration:
    \begin{enumerate}
        \item Guess a vector of \textcolor{red}{$\delta_i(s)$}
        \item Solve for \textcolor{darkgreen}{$D_1(s)$}
        \begin{align}
        \textcolor{darkgreen}{D_1(s)} = (\sum_i^{n(s)} [\textcolor{red}{\delta_i(s)} z_i(s)^{-\frac{1}{ \hat{\eta}}-1}U((\frac{\textcolor{red}{\delta_i(s)} z_i(s)^{-\frac{1}{\hat{\eta}}-1}}{\textcolor{darkgreen}{D_1(s)}})^{\frac{\hat{\eta}(1-\eta)}{\eta + \hat{\eta}}})]^{\frac{\hat{\eta}(1-\eta)}{\eta+\hat{\eta}}})^{\frac{\eta+\hat{\eta}  }{\hat{\eta}(1-\eta)}}
    \end{align}
        \item Calculate $x_i(s)$ and then $\mu_i(s)$.
    \item Use \textcolor{blue}{$\mu_i(s)$} to solve for \textcolor{melon}{$D_2(s)$}. 
    \begin{align}
         \textcolor{melon}{D_2(s)} = (\sum_i^{n(s)} [z_i(s)^{\frac{1}{\eta}-1}\Delta((\frac{\textcolor{blue}{\mu_i(s)} z_i(s)^{\frac{1}{\eta}-1}}{\textcolor{melon}{D_2(s)}})^{-\frac{\eta(\hat{\eta}+1)}{\eta+\hat{\eta}}})\textcolor{blue}{\mu_i(s)}]^{-\frac{\eta(\hat{\eta}+1)}{\eta+\hat{\eta}}})^{-\frac{\eta+\hat{\eta}}{\eta(\hat{\eta}+1)}}
    \end{align}
    \item Calculate $g_i(s)$ and $\delta_i(s)$
    \item Compare resulting $\delta_i(s)$ with initial guess 
    \item Iterate until $\delta_i(s)$ converges
    \end{enumerate}
\end{enumerate}

\section{Aggregation}


\section{Numerical Results}

% Compare markups with and without labor market power -> latter should exacerbate the former

% Look at distribution of skill premium like in the paper

% compare skill premium without frictions to get a measure of "efficient" inequality

% Aggregate welfare analysis



\section{Main Contributions}

% 1. Technical: if we can get a better/faster algorithm like in lecture to solve the equilibrium

% 2. Possible extensions
% - Include competitive sectors in addition to concentrated ones (unrealistic to say all sectors have market power)
% - crazy: should we add occupational choice... (?) 

% 3. TBH - the motivation of this paper is questionable. It is not clear at all why you need both product market power and labor market power to answer how the skill premium changes in the first place. To me, it makes more sense to do labor market power + skill heterogenity and thats it. Should we do this counterfactual exercise ??

% 4. Can the production function be simplified? so that the algorithm works better? perhaps firms can only employ low-skilled or high-skilled while high-skilled has a higher productivity -> task-based production functions? 

}


\end{document}
